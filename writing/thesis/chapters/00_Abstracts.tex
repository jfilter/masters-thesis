\chapter*{Acknowledgements}

I thank Professor Felix Naumann and his whole chair, especially Dr. Ralf Krestel, Julian Risch and Samuele Garda for their advise.

I thank my brother Julian Filter for commenting on drafts of this master's thesis.

I thank my family, the \textit{Studienstiftung} and the European Commission in the Erasmus program for supporting me financially during my time as a student.

I thank the whole Open-source Software community.

I thank all the people who fight for high-quality public education.

\chapter*{Abstract}

% Bitte nichts ändern.

Online newspapers close their comment section because they cannot cope with the sheer amount of user-generated content.
Natural-language processing allows to automatically classify news comments in order to efficiently support moderators.
Identifying hate speech is only a special case of comment classification and in this master's thesis we focus on classifying along any classification criteria, e.g., sentiment, off-topic, controversial.
In contrast to prior work, we consider the conversational context to be essential for understanding a comment's true meaning.
We introduce a preprocessing technique to prepend previous comments to training samples in order to apply state-of-the-art language-model-based text classification technique ULMFIT.
We conducted experiments on nine categories of the research dataset Yahoo News Annotated Comment Corpus.
With conversation-aware models, we increased the F\textsubscript{1 micro} and F\textsubscript{1 macro} scores on average by $1.53\%$ and $3.08\%$, respectively.
However, the differences to conversation-agnostic models vary among the  categories.
We achieved the biggest improvements when identifying whether a comment is off-topic or if it agrees or disagrees with other comments.
 
%Agreement:
%0.8
%4.4
%
%Disagreement:
%1.6
%1.4
%
%Off-topic
%2.1
%7


\chapter*{Zusammenfassung}

Online-Zeitungen schlie{\ss}en ihren Kommentarbereich, weil sie mit der schieren Menge an nutzergenerierten Inhalten nicht fertig werden.
Die maschinelle Sprachverarbeitung erm{\"o}glicht es, Zeitungs-User-Kommentare automatisch zu klassifizieren, um Moderatoren effizient zu unterst{\"u}tzen.
Die Identifizierung von Hasskommentaren ist nur ein Sonderfall der Kommentar- klassifizierung und in dieser Masterarbeit konzentrieren wir uns auf die Klassifizierung nach beliebigen Klassifizierungskriterien, wie z.B. Sentiment, Off-Topic, Kontroversivit{\"a}t.
Im Gegensatz zu fr{\"u}heren Arbeiten betrachten wir den Konversationskontext als wesentlich f{\"u}r das Verst{\"a}ndnis der wahren Bedeutung eines Kommentars.
Wir stellen eine Vorverarbeitungstechnik vor, um vorherige Kommentare an Lernbeispiele anzuf{\"u}gen, um die neueste sprachmodellbasierte Textklassifikationstechnik ULMFIT anzuwenden.
Wir haben Experimente an neun Kategorien des Forschungsdatensatzes \textit{Yahoo News Annotated Comment Corpus} durchgef{\"u}hrt.
Bei konversationsbewussten Modellen haben wir die Werte f{\"u}r F\textsubscript{1 micro} und F\textsubscript{1 macro} im Durchschnitt um $1,53\%$ bzw. $3,08\%$ erh{\"o}ht.
Die Unterschiede zu konversationsagnostischen Modellen sind jedoch je nach Kategorie unterschiedlich.
Wir haben die gr{\"o}{\ss}ten Verbesserungen erzielt, wenn es darum ging, festzustellen, ob ein Kommentar nicht zum Thema passt oder ob er mit anderen Kommentaren {\"u}bereinstimmt oder nicht.